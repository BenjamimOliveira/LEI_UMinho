\paragraph{}
A primeira ideia de \textit{platooning} foi introduzida pelo Norman Bel Geddes quando na feira mundial de 1939  apresentou o primeiro conceito de automóveis autónomos.
Em 1970 foi criado o projeto Aramis (\textit{Agencement en Rames Automatisées de Modules Indépendants dans les Stations}), sistema experimental \textit{personal rapid transit} (\textit{PRT}), desenvolvido em França era para ser instalado em Paris mas acabou por ser abandonado.  Este sistema tinha a característica única de que os veículos eram ligados eletronicamente (\textit{soft platooning}) e não mecanicamente (\textit{hard platooning}). Isto permite que os veículos funcionassem como comboios virtuais em áreas de alta densidade de transito, permitindo assim, um elevado rendimento.
Em 1980 foi criado o projeto \textit{Prometheus} (\textit{Program for European Traffic with Highest Efficiency and Unprecedented Safety}) pela Daimler-Benz AG em colaboração com universidades e empresas  de tecnologia na Europa. Este projeto tinha o objetivo de criar um sistema rodoviário inteligente que pudesse comunicar com os veículos que nele operassem, incorpora elementos de controlo de veículos que comunicam sem fios e usam inteligência artificial.
Em 1986 foi criado o projeto \textit{Partners for Advanced Transportation Technology} (\textit{PATH}) na Califórnia. Em 1994 criou um sistema automático de auto-estradas composto por um pelotão de quatro carros, utilizando uma posição de controlo longitudinal automatizada. Em 1997 aumentaram o numero de veículos para 8, e a sua ultima versão está a executar pelotões de três camiões que operam em intervalos de 4.2672 metros.
Em 1996 começou a primeira investigação de \textit{platooning} de camiões na Europa chamado de \textit{CHAUFFEUR}, financiado pela comissão europeia e partilhado com a indústria. O primeiro \textit{Class 8 truck tacktor} foi desenvolvido no âmbito do projeto que envolveu fabricantes de camiões como Daimler, Renault, e IVECO. Em 2000 foi criado o sucessor \textit{CHAUFFEUR II}.
Em 2004 a \textit{U.S. Defense Advanced Research Projects Agency (DARPA)} desafiou várias equipas, para que completassem um curso com veículos totalmente automatizados no deserto do Mojave e em 2007 numa área urbana.
Em 2005 o governo alemão financiou o projeto \textit{KONVOI}, no qual foram testados camiões totalmente automatizados em estradas publicas para avaliar as suas interações com o tráfego.
Em 2008 o Japão criou o projeto \textit{Energy ITS} para o \textit{platooning} de camiões motivado pela redução emissões de carbono e pela conservação de combustível. 
Um ano mais tarde em 2009 foi criado um novo conceito, o projeto \textit{SATRE}. Em que o condutor principal, o líder trataria da condução enquanto os outros veículos eram automatizados. Este projeto era liderado pela Volvo Cars e a Volvo Trucks.
Em 2010 foi criado o projeto \textit{Autonomous Mobility Applique System (AMAS)} pela Lockheed Martin 
a pedido da US DOD. O \textit{AMAS} fez testes durante mais de 88 513 920 metros. Os camiões totalmente automatizados do pelotão eram operados por soldados que assim estavam dispensados da condução para se concentrarem na sua missão militar.
Em 2013 foi criado o projeto \textit{EU COMPANION} com o objetivo de melhorar a eficiência do combustível e a segurança do transporte de mercadorias. 